Because of technolological advancements in materials, the gap between \emph{supercapacitors} and the conventional \emph{batteries} for large charge storage is reducing. Supercapacitors can charge really fast and have a much longer cycle life compared to batteries. Hence supercapacitors are increasing in importance and are slowly becoming a competitor to batteries. However, supercapacitors can hold about 10 times less energy than a Li-ion battery of the same size. The table below gives an overview of the important aspects of the supercapacitor and the battery we took into account during this project. As performance is always a trade-off we ended up combining a supercapacitor with a battery as will be discussed more into detail in Section \ref{sec:analysis}. 
\\
\\
%-----------------------Table ----------------------------------------
\begin{tabular*}{\textwidth}{@{\extracolsep{\fill}} |l|l|l|}
\hline
Characteristic & Supercapacitor & Li-ion Battery \\
\hline
Charge time & 1–10 seconds & 10–60 minutes \\
Cycle life & 1 million or 30,000h & 500 and higher \\
Cell voltage & 2.3 to 2.75 Volts & 3.6 to 3.7 Volts \\
Cost per Wh & Cost per Wh & $0.50-$1.00 (large system) \\
\hline
\end{tabular*}
\begin{center}
\textbf{ Table 1.} Comparison between a supercapacitor and a battery \cite{superbattery}
\end{center}
