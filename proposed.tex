%\begin{figure}[!h]
%\begin{center}
%\includegraphics[scale=0.50]{AD624.eps}
%\caption{Block Diagram of the instrumentation amplifier AD624.
%\end{center}
%\label{fig:design}
%\end{figure}

The Figure %~\ref{fig:design} shows the proposed design of the system. The system includes a transmitter and a receiver with transmission and receiving coil respectively. The wireless power transmission works on the principle of magnetic induction \cite{IAmp}.
The transmitter is powered up by a voltage source $V_i$ of 12 Volts capable of delivering 400 milli-Amps of $i_{in}$ current. $i_c$ is the constant current that is consumed by the transmitter circuitry and $i_s$ is the induction current which flows through the transmitter coil such that $i_{in} = i_c + i_s$ . $i_c$ is constant and depends on the transmitter inner circuitry power consumption, in our case $i_c = 100 milli-Amps $. $i_s$ depends on the distance between the two magnetically coupled coils, greater the distance smaller the $i_s$ will be. Another factor that $i_s$ could depend is on adding an iron core between the two coils, adding a core makes the magnetic coupling stronger and increases the $i_s$ which enhances an overall efficiency of the system.
The receiver circuit receives an induction voltage $V_r$, rectifies it through a rectifier containing a shotkey diode $D_r$ and a capacitor $C_r$. A shotkey diode is used in order to have a good frequency response at the range of $ 300-400 Khz$ the transmitter working frequency also shotkey has lower forward voltage drop. The rectified voltage is then fed to the voltage regulator that produces constant voltage $V_{reg} = 5 Volts$ .
